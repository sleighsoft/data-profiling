\documentclass{scrartcl}
\usepackage{amsmath, amssymb, amsthm}
\usepackage[ngerman]{babel}
\usepackage[utf8]{inputenc}
\usepackage[T1]{fontenc}
\usepackage{geometry}
\usepackage{color}
\usepackage{graphicx}
\usepackage{enumerate}
\usepackage[ruled, vlined, linesnumbered]{algorithm2e}


\setlength{\parindent}{0mm}

\begin{document}
\pagestyle{plain}

\section*{LighthouseUCC}

\section{Konzept}
LighthouseUCC ist ein von uns entwickelter Algorithmus und baut nicht direkt auf anderen Algorithmen auf.
\subsection{Genereller Ablauf}
Wir erzeugen PLIs für jede Spalte. 
Wir berechnen für jede Spalte die Anzahl distinct Werte als Score. 
Dann filtern wir alle Spalten heraus, die bereits Unique sind.
Die verbliebenen Kandidaten kommen in eine Liste.
Mainloop(Solange Kandidaten in unserer Liste vorhanden sind)
Wir sortieren die Kandidaten anhand ihres Scores. (PriorityQueue)
Wir wählen nun den Kandidaten mit dem höchsten Score und testen ihn auf Uniqueness.
Wenn er Unique ist:
- Füge ihn in die Unique Liste ein
- Entferne SuperSets dieses Kandidaten aus der Unique List
- Außerdem erzeugen wir direkte SubSets des Kandidaten und fügen diese in die Kandidaten Liste ein.
Sonst:
- Vom Kandidaten werden jetzt mit einzelnen Spalten Spaltenkombinationen gebildet.
- Dabei werden nur neue Spaltenkombinationen(keine Permutationen) berücksichtigt.
- Auch erzeugen wir keine Spaltenkombination für die wir bereits kleinere UCCs gefunden haben.
- Diese neuen Spaltenkombinationen sind neue Kandidaten in unserer Liste.

\subsection{NULL-Semantik}
$null \neq null$ \\
- potentiell hohe Wahrscheinlichkeit teil einer minimalen UCC zu sein\\
- durch hohen Score früh in unserem Algorithmus behandelt\\
$null = null$ \\
- potentiell niedrige Wahrscheinlichkeit teil einer minimalen UCC zu sein \\
- durch hohen Score spät in unserem Algorithmus behandelt
\section{Performance}
Unsere Messungen haben wir auf einem Windows PC mit i7 4500U 1,80GHz durchgeführt. Der Anwendung standen 2GB RAM zur Verfügung.\\

\begin{tabular}{c|c|c}
Datensatz & Number of UCCs & Time \\
\hline
ncvoter-1k.csv & 98 & 1s 206ms \\
adult.csv & none & 306ms \\
WDC\_astronomical.csv & 4 & 25ms \\
WDC\_satellites.csv & 4 & 44ms\\

\end{tabular}
\\
\\
Beim Testdurchlauf unseres Algorithmus auf dem Datenset ncvoter.csv(958MB) tritt ein Out-of-Heap-Fehler auf.\\

\end{document}
